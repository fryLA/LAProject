\documentclass[12pt]{book}
\usepackage[a4paper,left=1cm,right=1cm,top=3cm,bottom=4cm,bindingoffset=5mm]{geometry}
\usepackage{fancyhdr}
%\usepackage{lscape} %für Querfromat
\usepackage{cancel} %um was in durchzustreichen
\usepackage{paralist} %um in enumerate einfach andere Sachen anzugeben
\usepackage{amsmath}
\usepackage{amssymb}
\usepackage{amsthm}
\usepackage{subcaption}
\usepackage{todonotes}
\usepackage[ngerman]{babel}
\usepackage[utf8]{inputenc}
%\usepackage{MnSymbol}
\usepackage{mathabx} % damit man Doppelklammern haben kann
\setlength{\parindent}{0em}  %verhindert Einrücken nach Absatz
\usepackage{xargs} %
\usepackage{tikz} %zum zeichnen
\usepackage{expdlist} % um den Zeilenabstand in description kleiner zu machen


\newcommand\tab[1][1cm]{\hspace*{#1}}

%\setmathfont[range={\lsem,\rsem}]{XITS Math}

% Meine Schrift, die mathe kann
\usepackage[math]{iwona}
\usepackage[T1]{fontenc}


\pagestyle{fancy}
\fancyhf{}
\lhead{Graph Drawing - Layer Assignment}
\rhead{ Thomas Dost, Jonas Lange, Francesca Rybicki}

\begin{document}
 
\section*{Subject}

We aim to visualize the \textit{Layer Assignment} phase of the \textit{Layered} layout algorithm step by step.

\todo[inline]{write summarizing stuff here, add source of layered paper}

\section*{Our Approach}

here should be some general information about things we do.

\begin{enumerate}
    \item first we parse
    \item then we apply the algorithm
    \item then we visualize
\end{enumerate}

\todo[inline]{write general information on your topic here, e.g what is you input, what is yout output, any methods you use.. but don't go into detail yet. this will happen in your very own chapter }

\section*{Approach in detail}

\todo[inline]{write a subchapter about your topic. go into details, if there is something you really want to mention, because its cool or so.. especially try to use stuff from lecture (sources, technical terms, etc. ) if you want to give reasons for your solution. 

Further: explain the input and output formats of your special topic -> iLearn has some MUSTs listed here, be sure you considered those.}





\subsection*{Parsing}
Die Parsing-Klasse ist dafür zustädnig das eigens erstelle Textformate einzulesen
und in einen ElkGraph umzuwandeln

\subsubsection*{Syntax}
Eine Node hinzufügen:
\newline
\tab[0.5cm]  node \_$<$node name$>$
\newline
\newline
Eine Edge hinzufügen:\newline
\tab[0.5cm]  edge\_$<$start node$>$ \_$<$end node$>$
\newline
Wobei \_ für beliebig viele Whitespaces steht,
weiter muss jeder Knoten der von einer edge benutzt wird vorher mit dem \textit{node} Keyword eingeführt werden

\subsubsection*{Methoden}
Parser.parse(<filepath>)\newline
Return: null falls ein fehler auftrat, sonst ein ElkGraph

\subsubsection*{How to use the parser}
try 
\{\newline
ElkNode testGraph = Parser.parse("testGraphs/testfile.txt");
\newline
\}
\newline
catch (Exception e)
\newline
\{
\newline
  e.printStackTrace(); 
\newline
\}


\subsubsection*{Example}
node n1 \newline
node n2 \newline
node n3 \newline
\newline
\newline
edge n1 n3\newline 
edge n2 n3 \newline












\subsection*{Layer Assignment Algorithm}
Die Klasse LayerAssignment enthällt eine Reihe hilfreicher Funktionen, die Wichtigste ist allerdings die Methode assignLayers. assignLayers nimmt einen ElkGraphen, führt das Layerassignment an ihm durch und gibt eine Bearbeitungshistorie in Form einer ArrayList vom typen MyGraph zurück. Wobei MyGraph eine von uns entwickelte vereinfachte Graphenstruktur ist. Jeder MyGraph in der Liste stellt also einen "snapshot" aus dem Layerassingments des Graphen dar.
Durch Seiteneffekte beinflusst die Methode auch den eingegebenen ElkGraphen, so dass er am Ende der Methode als vollständig gelayerter Graph vorliegt. Dazu haben wir jedem Knoten 3 Propertys gegeben:
 LAYER: Ein Integer wert größer oder gleich 1, der das Layer angibt, in dem sich der Knoten befindet

IS\_DUMMY: Ein boolscher wert, der True ist, wenn es sich bei dem Knoten um einen von uns eingefügten Dummyknoten handelt. Zu beachten ist, dass auch Edges die Property IS\_DUMMY haben.

POSITION\_IN\_LAYER: Gibt an, an welcher Stelle im Layer sich der Knoten befindet, diese Property muss in der Crossingminimisation optimiert werden.

Beim einfügen der Dummyknoten geht der Algorithmus wie folgt vor:
Wenn eine Kante k ein Layer übersprngt, fügen wir für jedes Layer zwischen den Layern des Startknotens von k und des Endknotens von k einen neuen Dummyknoten in dem jeweiligen Layer ein. Nun fügen wir Dummykanten so ein, dass von jedem Dummyknoten eine Dummykante zum Dummyknoten im nächsthöheren Layer führt. Desweiteren soll eine Dummykante vom höchsten Dummyknoten zum Zielknotne von k führen. Zuletzt wird noch der Zielknoten von k auf den Dummyknoten im Layer direkt über dem Startknoten von k gesetzt.

(Hier vlt zeichnung)


MyGraph ist eine vereinfachte Graphenstruktur. Sie enthällt nur eine Liste von Knoten und eine Liste von Kanten.  















  \section*{Visualisierung}
Die GUI wurde mittels \textit{Java Swing} implementiert und nutzt ein \textit{JSplitPane}, um die textuelle Darstellung eines Graphen gleichzeitig mit der Visualisierung des auf diesen Graphen angewandten Algorithmus anzuzeigen. Der Graph ist hierbei in textueller Form editierbar, so dass der Benutzer komfortabel Änderungen im Graphen vornehmen kann.
\newline 
Da wir im Algorithmus keinen \textit{longest path} berechnen, um die Anzahl der nötigen Layer zu bestimmen, können wir in jedem Schritt alle Quellknoten einem Layer zuordnen. Quellknoten sind hierbei Knoten ohne eingehende Kanten, deren Startknoten noch keinem Layer zugewiesen wurden). Die tatsächliche Anzahl der benötigten Layer wird über die Länge der SimpleGraph Liste berechnet, in der noch keine Dummyknoten eingesetzt wurden. Diese Liste erhält die  Visualisierung bevor sie gestartet wird.
\newline 
In welchem Schritt sich der Algorithmus gerade befindet, wird oberhalb des visualisierten Graphen angezeigt. Weiterhin wird jeder Knoten dessen Position sich ändert rötlich eingefärbt. Jede Kante, die durch einen Dummyknoten und zwei neue Kanten ersetzt wird, wird kurz bevor sie ausgeblendet wird rötlich hervorgehoben, um jeden Schritt der Berechnung nachvollziehbar zu machen. Um DummyKnoten erkennbar zu machen, werden sie rund und blasser als normale Knoten dargestellt.

\subsection*{Öffnen eines Graphen}
Momentan gibt es einen Unterordner \textit{testGraphs} im Projekt, der einige Testgraphen enthält. Der Inhalt dieses Ordners wird dem Benutzer im File Dialog angezeigt, wenn er auf \textit{Load File} in der Applikation klickt. Dies war zu Testzwecken sinnvoll und wurde bisher nicht angepasst.


\subsection*{Speichern eines Graphen}
Momentan gibt es einen Unterordner \textit{savedImages} im Projekt. Der Inhalt dieses Ordners wird dem Benutzer im File Dialog angezeigt, wenn er über den Rechtsklick in der Applikation und dann auf \textit{Save Image} klickt. Um nun die aktuelle Ansicht als Bild zu speichert, muss ein Name angegeben werden. \textbf{Achtung! Keinen Dateityp angeben!} Das Bildt wird als \textit{.png} im entsprechenden Ordner gespeichert.



\subsection*{User Interface}
\begin{enumerate}
    \item[Slider und Button:] Der \textit{Step Size} Slider startet aus Ästhetikgründen für die Beschriftungen bei 0. Wird hier die 0 ausgewählt, wird kein Update an der Visualisierung vorgenommen. Die Obergrenze ist die Anzahl der möglichen  Berechnungsschritte. Befindet sich die Berechnung bereits in einem fortgeschrittenen Berechnungsschritt, wird beim Modifizieren des Sliders überprüft ob die Maximale Anzahl der Berechnungsschritte nicht überschritten wird. Dieser Slider bietet also eine zusätzliche Möglichkeit zum \textit{Speed} Slider die Animation des Algorithmus zu beschleunigen. Um die Anzahl der Schritte genau angeben und ablesen zu können wurde außerdem ein Textfeld hinzugefügt, welches mit dem Slider Synchronisiert ist. Der Slider kann nicht verwendet werden während die Animation des Algorithmus, die durch den \textit{Play} Button gestartet wurde, läuft. 
    
    Der im \textit{Speed} Slider angegebene Wert definiert, wie viele Schritte die Knoten zwischen ihrer Start und Zielposition zurücklegen sollen. Ist der Wert des Sliders hoch, so erreichen die Knoten ihre Zielposition in weniger Schritten und umgekehrt. Die Geschwindigkeit ist über die Anzahl der Berechnungszyklen definiert, da wir erreichen wollen, dass die  Knoten alle zum (fast) gleichen Zeitpunkt ihre Zielposition erreichen, unabhängig von der Strecke, die sie dafür zurücklegen müssen. Obere und Untere Grenzen wurden so gewählt, dass die Animation flüssig läuft und nachvollziehbar bleibt.
    
    Button und Slider verfügen über Tooltips, um die Bedienung zu vereinfachen.
    
    \item[Toolbar:] Die Elemente für die Justierung und Steuerung der Visualisierung sind in einer Toolbar zusammengefasst. Das hat den Vorteil, dass man diese per \textit{drag and drop} aus dem Fester heraus- und auch wieder hineinziehen kann.
    
\end{enumerate}

\subsection*{Animation}



Die Animation wird über Updates und einen Timer ausgeführt. Bevor der Timer gestartet wird erhält jeder Knoten eine neue Zielposition. Wird der Timer gestartet, bewegt sich jeder Knoten auf seine Zielposition zu bis er diese erreicht hat. Hierbei wird eine Updatefunktion für die Positionierung der Knoten entsprechend häufig durch den Timerevents ausgelöst. In jedem Timerevent wird die View neu gezeichnet. Das heißt die Kanten, deren Positionen von ihren Start- und Zielknoten abhängen, werden ebenfalls mit den Knoten Bewegt, was zu einer flüssig ablaufenden Animation führt.

Wurde der \textit{Play} Button gedrückt, so wird ein Thread gestartet, der für jeden Schritt die Knotenpositionen erneuert und den Timer für die Updatefunktion startet. Dieser Thread wartet darauf, dass der Timer nicht mehr aktiv ist bevor er den Schritt der Berechnung ausführt.

Geht man jedoch Schrittweise durch den Algorithmus, indem mal den \textit{Forward} oder \textit{Backward} Button drückt, so kann man den nächsten Schritt bereits ausführen, während die Animation noch läuft. An dieser Stelle erschien es uns aus rechentechnischen Gründen nicht sinnvoll per buisy waiting o.Ä. das Ende der Animation abzuwarten, bevor die Button wieder ansprechbar sind.






\section*{Schedule}


\begin{tabular}{|p{2cm}|*3{p{4.5cm}|}}
   
    \hline
    Date & Thomas & Jonas & Francesca \\
     \hline
     \hline
      10.6.18 & - & - & Window mockup (no functionality) \\
       \hline
      11.6.18 & \multicolumn{3}{|c|}{Fill Schedule with your assumptions!!!!} \\
     \hline
    ? & Parsing finished & - & - \\
    \hline
     ? & - & First approach for Layer assignment & - \\
    \hline
     ? & - & - & Animations applied to LA algorithm \\
    \hline
   19.6.18 & \multicolumn{3}{|c|}{First Deadline/Demo} \\
    \hline
   ? &? & ?&? \\
   \hline
   7.7.18 & \multicolumn{3}{|c|}{Complete Documentation} \\
    \hline
     9.7.18 &\multicolumn{3}{|c|}{Last Checks?} \\
     \hline
     10.7.18 &\multicolumn{3}{|c|}{Finished} \\
    
    \hline
\end{tabular}


\todo[inline]{Alle Daten im schedule sollten bis zur Zwischendemo feststehen, damit wir den dort verwenden können. Also tragt gerne viel Zeug hier ein}

\section*{Technical Details}

\todo[inline]{things like class diagrams maybe?}


\end{document}






