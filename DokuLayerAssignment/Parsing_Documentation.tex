



\subsection*{Parsing}
Die Parsing-Klasse ist dafür zustädnig das eigens erstelle Textformate einzulesen
und in einen ElkGraph umzuwandeln

\subsubsection*{Syntax}
Eine Node hinzufügen:
\newline
\tab[0.5cm]  node \_$<$node name$>$
\newline
\newline
Eine Edge hinzufügen:\newline
\tab[0.5cm]  edge\_$<$start node$>$ \_$<$end node$>$
\newline
Wobei \_ für beliebig viele Whitespaces steht,
weiter muss jeder Knoten der von einer edge benutzt wird vorher mit dem \textit{node} Keyword eingeführt werden

\subsubsection*{Methoden}
Parser.parse(<filepath>)
Return: null falls ein fehler auftrat, sonst ein ElkGraph

\subsubsection*{How to use the parser}
try 
\{\newline
ElkNode testGraph = Parser.parse("testGraphs/testfile.txt");
\newline
\}
\newline
catch (Exception e)
\newline
\{
\newline
  e.printStackTrace(); 
\newline
\}


\subsubsection*{Example}
node n1 \newline
node n2 \newline
node n3 \newline
\newline
\newline
edge n1 n3\newline 
edge n2 n3 \newline









