


\subsection*{Parsing}
Die Parsing-Klasse ist dafür zustädnig das eigens erstelle Textformate einzulesen
und in einen ElkGraph umzuwandeln

\subsubsection*{Syntax}
Eine Node hinzufügen:
  node_<node name>
  
Eine Edge hinzufügen:
  edge_<start node>_<end node>
 
Wobei _ für beliebig viele Whitespaces steht,
weiter muss jeder Knoten der von einer edge benutzt wird vorher mit dem \textit{node} Keyword eingeführt werden

\subsubsection*{Example}
node n1
node n2
node n3


edge n1 n3
edge n2 n3 












