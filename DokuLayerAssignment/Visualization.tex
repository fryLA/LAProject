  \section*{Visualisierung}
Die GUI wurde mittels \textit{Java Swing} implementiert und nutzt ein \textit{JSplitPane}, um die textuelle Darstellung eines Graphen gleichzeitig mit der Visualisierung des auf diesen Graphen angewandten Algorithmus anzuzeigen. Der Graph ist hierbei in textueller Form editierbar, so dass der Benutzer komfortabel Änderungen im Graphen vornehmen kann.
\newline 
Da wir im Algorithmus keinen \textit{longest path} berechnen, um die Anzahl der nötigen Layer zu bestimmen, können wir in jedem Schritt alle Quellknoten einem Layer zuordnen. Quellknoten sind hierbei Knoten ohne eingehende Kanten, deren Startknoten noch keinem Layer zugewiesen wurden). Die tatsächliche Anzahl der benötigten Layer wird über die Länge der SimpleGraph Liste berechnet, in der noch keine Dummyknoten eingesetzt wurden. Diese Liste erhält die  Visualisierung bevor sie gestartet wird.
\newline 
In welchem Schritt sich der Algorithmus gerade befindet, wird oberhalb des visualisierten Graphen angezeigt. Weiterhin wird jeder Knoten dessen Position sich ändert rötlich eingefärbt. Jede Kante, die durch einen Dummyknoten und zwei neue Kanten ersetzt wird, wird kurz bevor sie ausgeblendet wird rötlich hervorgehoben, um jeden Schritt der Berechnung nachvollziehbar zu machen. Um DummyKnoten erkennbar zu machen, werden sie rund und blasser als normale Knoten dargestellt.

\subsection*{Öffnen eines Graphen}
Momentan gibt es einen Unterordner \textit{testGraphs} im Projekt, der einige Testgraphen enthält. Der Inhalt dieses Ordners wird dem Benutzer im File Dialog angezeigt, wenn er auf \textit{Load File} in der Applikation klickt. Dies war zu Testzwecken sinnvoll und wurde bisher nicht angepasst.


\subsection*{Speichern eines Graphen}
Momentan gibt es einen Unterordner \textit{savedImages} im Projekt. Der Inhalt dieses Ordners wird dem Benutzer im File Dialog angezeigt, wenn er über den Rechtsklick in der Applikation und dann auf \textit{Save Image} klickt. Um nun die aktuelle Ansicht als Bild zu speichert, muss ein Name angegeben werden. \textbf{Achtung! Keinen Dateityp angeben!} Das Bildt wird als \textit{.png} im entsprechenden Ordner gespeichert.



\subsection*{User Interface}
\begin{enumerate}
    \item[Die View] Aufteilung
    \item[Slider und Button] Der \textit{Step Size} Slider startet aus Ästhetikgründen für die Beschriftungen bei 0. Wird hier die 0 ausgewählt, wird kein Update an der Visualisierung vorgenommen. Die Obergrenze ist die Anzahl der möglichen  Berechnungsschritte. Befindet sich die Berechnung bereits in einem fortgeschrittenen Berechnungsschritt, wird beim Modifizieren des Sliders überprüft ob die Maximale Anzahl der Berechnungsschritte nicht überschritten wird. Dieser Slider bietet also eine zusätzliche Möglichkeit zum \textit{Speed} Slider die Animation des Algorithmus zu beschleunigen. Um die Anzahl der Schritte genau angeben und ablesen zu können wurde außerdem ein Textfeld hinzugefügt, welches mit dem Slider Synchronisiert ist. Der Slider kann nicht verwendet werden während die Animation des Algorithmus, die durch den \textit{Play} Button gestartet wurde, läuft. 
    
    Der im \textit{Speed} Slider angegebene Wert definiert, wie viele Schritte die Knoten zwischen ihrer Start und Zielposition zurücklegen sollen. Ist der Wert des Sliders hoch, so erreichen die Knoten ihre Zielposition in weniger Schritten und umgekehrt. Die Geschwindigkeit ist über die Anzahl der Berechnungszyklen definiert, da wir erreichen wollen, dass die  Knoten alle zum (fast) gleichen Zeitpunkt ihre Zielposition erreichen, unabhängig von der Strecke, die sie dafür zurücklegen müssen. Obere und Untere Grenzen wurden so gewählt, dass die Animation flüssig läuft und nachvollziehbar bleibt.
    
    Button und Slider verfügen über Tooltips, um die Bedienung zu vereinfachen.
    
    \item[Toolbar] Die Elemente für die Justierung und Steuerung der Visualisierung sind in einer Toolbar zusammengefasst. Das hat den Vorteil, dass man diese per \textit{drag and drop} aus dem Fester heraus- und auch wieder hineinziehen kann.
    
\end{enumerate}


